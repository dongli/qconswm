\documentclass{ctexart}

\usepackage{amsmath}
\usepackage[margin=1in]{geometry}
\usepackage{xcolor}
\usepackage{hyperref}

\title{基于平方守恒的全球浅水模型}
\date{2017年9月21日}

\begin{document}

\maketitle

\section{出发方程}

\begin{align}
	\frac{\partial U}{\partial t} & = - \frac{1}{2 a \cos{\varphi}} \left( 2 \frac{\partial u U}{\partial \lambda} - U \frac{\partial u}{\partial \lambda} + 2 \frac{\partial v \cos{\varphi} U}{\partial \varphi} - U \frac{\partial v \cos{\varphi}}{\partial \varphi} \right) + f V + \frac{u \tan{\varphi}}{a} V - \frac{H}{a \cos{\varphi}} \frac{\partial \phi + \phi_s}{\partial \lambda} \\
  \frac{\partial V}{\partial t} & = - \frac{1}{2 a \cos{\varphi}} \left( 2 \frac{\partial u V}{\partial \lambda} - V \frac{\partial u}{\partial \lambda} + 2 \frac{\partial v \cos{\varphi} V}{\partial \varphi} - V \frac{\partial v \cos{\varphi}}{\partial \varphi} \right) - f U - \frac{u \tan{\varphi}}{a} U - \frac{H}{a} \frac{\partial \phi + \phi_s}{\partial \varphi} \\
  \frac{\partial \phi}{\partial t} & = - \frac{1}{a \cos{\varphi}} \left( \frac{\partial H U}{\partial \lambda} + \frac{\partial H V \cos{\varphi}}{\partial \varphi} \right)
\end{align}
其中$\varphi$是位势厚度,$\varphi_s$是地表位势抬升,$H = \sqrt{\varphi}$,$U = H u$,$V = H v$,$a$是球面半径,$\lambda$是经度,$\phi$是纬度。注意前两式中的平流项采用的是
\begin{align}
  2 \frac{\partial u \cdot}{\partial \lambda} - \cdot \frac{\partial u}{\partial \lambda} + 2 \frac{\partial v \cos{\varphi} \cdot}{\partial \varphi} - \cdot \frac{\partial v \cos{\varphi}}{\partial \varphi}
\end{align}
而不是
\begin{align}
  \frac{\partial u \cdot}{\partial \lambda} + u \frac{\partial \cdot}{\partial \lambda} + \frac{\partial v \cos{\varphi} \cdot}{\partial \varphi} + v \cos{\varphi} \frac{\partial \cdot}{\partial \varphi}
\end{align}
这是由于虽然两式在连续情形下等价,但第二式在Arakawa-C网格下进行差分离散无法满足反对称性要求,而第一式可以满足此要求,具体推导参见附录\ref{app:平流项的反对称性}。

将待求变量组成向量形式
\begin{align}
  F = \left[
    \begin{array}{c}
      U \\
      V \\
      \varphi
    \end{array}
  \right]
\end{align}
方程右端项整理为包含平流项与科氏力的慢过程算子
\begin{align}
  L_s = \left[
    \begin{array}{ccc}
      L_a^u & f & 0 \\
      - f & L_a^v & 0 \\
      0 & 0 & L_a^\varphi
    \end{array}
  \right]
\end{align}
其中
\begin{align}
  L_a^u & = - \frac{1}{2 a \cos{\varphi}} \left( \frac{\partial u \cdot}{\partial \lambda} + u \frac{\partial \cdot}{\partial \lambda} + \frac{\partial v \cos{\varphi} \cdot}{\partial \varphi} + v \cos{\varphi} \frac{\partial \cdot}{\partial \varphi}  \right) \\
  L_a^v & = - \frac{1}{2 a \cos{\varphi}} \left( \frac{\partial u \cdot}{\partial \lambda} + u \frac{\partial \cdot}{\partial \lambda} + \frac{\partial v \cos{\varphi} \cdot}{\partial \varphi} + v \cos{\varphi} \frac{\partial \cdot}{\partial \varphi} \right) \\
  L_a^\varphi & = - \frac{1}{2 a \cos{\varphi}} \left( \frac{\partial u \cdot}{\partial \lambda} + \frac{\partial v \cos{\varphi} \cdot}{\partial \varphi} \right)
\end{align}
包含重力项(梯度力项)的快过程算子
\begin{align}
  L_f = \left[
    \begin{array}{ccc}
      0 & 0 & L_g^u \\
      0 & 0 & L_g^v \\
      0 & 0 & 0
    \end{array}
  \right]
\end{align}
其中
\begin{align}
  L_g^u & = - \frac{H}{a \cos{\varphi}} \frac{\partial \cdot + \phi_s}{\partial \lambda} \\
  L_g^v & = - \frac{H}{a} \frac{\partial \cdot + \phi_s}{\partial \varphi}
\end{align}
可见算子$L_s$和$L_f$是非线性算子。

\section{空间离散}
采用等间距的经纬网格,变量布置采用Arakawa-C,即纬向风分量位于纬向半格点$\left( i + \frac{1}{2}, j \right)$,经向风分量位于经向半格点$\left( i, j + \frac{1}{2} \right)$,位势厚度等量位于整网格点$\left( i, j \right)$。
\begin{align}
  H_{i+\frac{1}{2},j} & = \frac{H_{i,j} + H_{i+1,j}}{2} & H_{i,j+\frac{1}{2}} & = \frac{H_{i,j} + H_{i,j+1}}{2} \nonumber \\
  U_{i+\frac{1}{2},j} & = H_{i+\frac{1}{2},j} u_{i+\frac{1}{2},j} & V_{i,j+\frac{1}{2}} & = H_{i,j+\frac{1}{2}} v_{i,j+\frac{1}{2}} & \phi_{i,j} \nonumber
\end{align}
对出发方程右端微分项采用中央差离散,但是需要注意差分时应该采用一个格距,即需要将不同位置的变量平均到差分位置上。

\subsection{变量$U$的作用项}
\begin{align}
  \left( \frac{\partial u U}{\partial \lambda} \right)_{i+\frac{1}{2},j} & \approx \frac{1}{4 \Delta{\lambda}} \left[ \underbrace{\left( u_{i+\frac{1}{2},j} + u_{i+\frac{3}{2},j} \right)}_{u_{i+1,j}} \underbrace{\left( U_{i+\frac{1}{2},j} + U_{i+\frac{3}{2},j} \right)}_{U_{i+1,j}} - \underbrace{\left( u_{i+\frac{1}{2},j} + u_{i-\frac{1}{2},j} \right)}_{u_{i,j}} \underbrace{\left( U_{i+\frac{1}{2},j} + U_{i-\frac{1}{2},j} \right)}_{U_{i,j}} \right] \nonumber \\
  \left( U \frac{\partial u}{\partial \lambda} \right)_{i+\frac{1}{2},j} & \approx \frac{U_{i+\frac{1}{2},j}}{2 \Delta{\lambda}} \left[ \underbrace{\left( u_{i+\frac{1}{2},j} + u_{i+\frac{3}{2},j} \right)}_{u_{i+1,j}} - \underbrace{\left( u_{i+\frac{1}{2},j} + u_{i-\frac{1}{2},j} \right)}_{u_{i,j}} \right] = \frac{U_{i+\frac{1}{2},j}}{2 \Delta{\lambda}} \left[ u_{i+\frac{3}{2},j} - u_{i-\frac{1}{2},j} \right] \nonumber \\
  \left( \frac{\partial v^* U}{\partial \varphi} \right)_{i+\frac{1}{2},j} & \approx \frac{1}{4 \Delta{\varphi}} \left[ \underbrace{\left( v_{i,j+\frac{1}{2}}^* + v_{i+1,j+\frac{1}{2}}^* \right)}_{v_{i+\frac{1}{2},j+\frac{1}{2}}^*} \underbrace{\left( U_{i+\frac{1}{2},j} + U_{i+\frac{1}{2},j+1} \right)}_{U_{i+\frac{1}{2},j+\frac{1}{2}}} - \underbrace{\left( v_{i,j-\frac{1}{2}}^* + v_{i+1,j-\frac{1}{2}}^* \right)}_{v_{i+\frac{1}{2},j-\frac{1}{2}}^*} \underbrace{\left( U_{i+\frac{1}{2},j-1} + U_{i+\frac{1}{2},j} \right)}_{U_{i+\frac{1}{2},j-\frac{1}{2}}} \right] \nonumber \\
  \left( U \frac{\partial v^*}{\partial \varphi} \right)_{i+\frac{1}{2},j} & \approx \frac{U_{i+\frac{1}{2},j}}{2 \Delta{\varphi}} \left[ \underbrace{\left( v_{i,j+\frac{1}{2}}^* + v_{i+1,j+\frac{1}{2}}^* \right)}_{v_{i+\frac{1}{2},j+\frac{1}{2}}^*} - \underbrace{\left( v_{i,j-\frac{1}{2}}^* + v_{i+1,j-\frac{1}{2}}^* \right)}_{v_{i+\frac{1}{2},j-\frac{1}{2}}^*} \right] \nonumber \\
  \left( H \frac{\partial \phi^*}{\partial \lambda} \right)_{i+\frac{1}{2},j} & = \frac{H_{i,j} + H_{i+1,j}}{2 \Delta{\lambda}} \left( \phi_{i+1,j}^* - \phi_{i,j}^* \right) \nonumber
\end{align}
其中$v^* = v \cos{\varphi}$和$\phi^* = \phi + \phi_s$为方便推导做的简记。

\begin{align}
  \left( 2 \frac{\partial u U}{\partial \lambda} - U \frac{\partial u}{\partial \lambda} \right)_{i+\frac{1}{2},j} & = \left\{
    \begin{array}{ll}
      \frac{1}{2 \Delta{\lambda}} \left[ \left( u_{i+\frac{1}{2},j} + u_{i+\frac{3}{2},j} \right) U_{i+\frac{3}{2},j} \right. & \\
      \quad \left. - \left( u_{i+\frac{1}{2},j} + u_{i-\frac{1}{2},j} \right) U_{i-\frac{1}{2},j} \right] & j = [2, 3, \dots, N_\varphi - 1] \\
      0 & j = [1, N_\varphi]
    \end{array}
  \right. \\
  \left( 2 \frac{\partial v^* U}{\partial \varphi} - U \frac{\partial v^*}{\partial \varphi} \right)_{i+\frac{1}{2},j} & = \left\{
    \begin{array}{ll}
      \frac{1}{2 \Delta{\varphi}} \left[ \left( v_{i,j+\frac{1}{2}}^* + v_{i+1,j+\frac{1}{2}}^* \right) U_{i+\frac{1}{2},j+1} \right. & \\
      \quad \left. - \left( v_{i,j-\frac{1}{2}}^* + v_{i+1,j-\frac{1}{2}}^* \right) U_{i+\frac{1}{2},j-1} \right] & j = [2, 3, \dots, N_\varphi - 1] \\
      0 & j = [1, N_\varphi]
    \end{array}
  \right. \\
  \left(f V\right)_{i+\frac{1}{2},j} & = \left\{
    \begin{array}{ll}
      \frac{1}{4} \left[ f_{j+\frac{1}{2}} \left( V_{i,j+\frac{1}{2}} + V_{i+1,j+\frac{1}{2}} \right) \right. \\
      \left. + f_{j-\frac{1}{2}} \left( V_{i,j-\frac{1}{2}} + V_{i+1,j-\frac{1}{2}} \right) \right] & j = [2, 3, \dots, N_\varphi - 1] \\
      0 & j = [1, N_\varphi]
    \end{array}
  \right. \\
  \left(\frac{u \tan{\varphi}}{a} V\right)_{i+\frac{1}{2},j} & \approx \left\{
    \begin{array}{ll}
      \frac{u_{i+\frac{1}{2},j}^*}{4 a} \left[ V_{i,j+\frac{1}{2}} + V_{i+1,j+\frac{1}{2}} + V_{i,j-\frac{1}{2}} + V_{i+1,j-\frac{1}{2}} \right] & j = [2, 3, \dots, N_\varphi - 1] \\
      0 & j = [1, N_\varphi]
    \end{array}
  \right. \\
  \left( H \frac{\partial \phi^*}{\partial \lambda} \right)_{i+\frac{1}{2},j} & = \left\{
    \begin{array}{ll}
      \frac{H_{i,j} + H_{i+1,j}}{2 \Delta{\lambda}} \left( \phi_{i+1,j}^* - \phi_{i,j}^* \right) & j = [2, 3, \dots, N_\varphi - 1] \\
      0 & j = [1, N_\varphi]
    \end{array}
  \right.
\end{align}

\subsection{变量$V$的作用项}
\begin{align}
  \left( \frac{\partial u V}{\partial \lambda} \right)_{i,j+\frac{1}{2}} & \approx \frac{1}{4 \Delta{\lambda}} \left[ \underbrace{\left( u_{i+\frac{1}{2},j} + u_{i+\frac{1}{2},j+1} \right)}_{u_{i+\frac{1}{2},j+\frac{1}{2}}} \underbrace{\left( V_{i,j+\frac{1}{2}} + V_{i+1,j+\frac{1}{2}} \right)}_{V_{i+\frac{1}{2},j+\frac{1}{2}}} - \underbrace{\left( u_{i-\frac{1}{2},j} + u_{i-\frac{1}{2},j+1} \right)}_{u_{i-\frac{1}{2},j+\frac{1}{2}}} \underbrace{\left( V_{i-1,j+\frac{1}{2}} + V_{i,j+\frac{1}{2}} \right)}_{V_{i-\frac{1}{2},j+\frac{1}{2}}} \right] \nonumber \\
  \left( V \frac{\partial u}{\partial \lambda} \right)_{i,j+\frac{1}{2}} & \approx \frac{V_{i,j+\frac{1}{2}}}{2 \Delta{\lambda}} \left[ \underbrace{\left( u_{i+\frac{1}{2},j} + u_{i+\frac{1}{2},j+1} \right)}_{u_{i+\frac{1}{2},j+\frac{1}{2}}} - \underbrace{\left( u_{i-\frac{1}{2},j} + u_{i-\frac{1}{2},j+1} \right)}_{u_{i-\frac{1}{2},j+\frac{1}{2}}} \right] \nonumber \\
  \left( \frac{\partial v^* V}{\partial \varphi} \right)_{i,j+\frac{1}{2}} & \approx \frac{1}{4 \Delta{\varphi}} \left[ \underbrace{\left( v_{i,j+\frac{1}{2}}^* + v_{i,j+\frac{3}{2}}^* \right)}_{v_{i,j+1}^*} \underbrace{\left( V_{i,j+\frac{1}{2}} + V_{i,j+\frac{3}{2}} \right)}_{V_{i,j+1}} - \underbrace{\left( v_{i,j-\frac{1}{2}}^* + v_{i,j+\frac{1}{2}}^* \right)}_{v_{i,j}^*} + \underbrace{\left( V_{i,j-\frac{1}{2}} + V_{i,j+\frac{1}{2}} \right)}_{V_{i,j}} \right] \nonumber \\
  \left( V \frac{\partial v^*}{\partial \varphi} \right)_{i,j+\frac{1}{2}} & \approx \frac{V_{i,j+\frac{1}{2}}}{2 \Delta{\varphi}} \left[ \underbrace{\left( v_{i,j+\frac{1}{2}}^* + v_{i,j+\frac{3}{2}}^* \right)}_{v_{i,j+1}^*} - \underbrace{\left( v_{i,j-\frac{1}{2}}^* + v_{i,j+\frac{1}{2}}^* \right)}_{v_{i,j}^*} \right] \nonumber \\
  \left( H \frac{\partial \phi^*}{\partial \varphi} \right)_{i,j+\frac{1}{2}} & \approx \frac{H_{i,j} + H_{i,j+1}}{2 \Delta{\varphi}} \left( \phi_{i,j+1}^* - \phi_{i,j}^* \right)
\end{align}

\begin{align}
  \left( 2 \frac{\partial u V}{\partial \lambda} - V \frac{\partial u}{\partial \lambda} \right)_{i,j+\frac{1}{2}} & \approx \frac{1}{2 \Delta{\lambda}} \left[ \left( u_{i+\frac{1}{2},j} + u_{i+\frac{1}{2},j+1} \right) V_{i+1,j+\frac{1}{2}}  - \left( u_{i-\frac{1}{2},j} + u_{i-\frac{1}{2},j+1} \right) V_{i-1,j+\frac{1}{2}} \right] \nonumber \\
  & \quad\quad\quad\quad\quad j = [1,2,\dots,N_\varphi - 1] \\
  \left( 2 \frac{\partial v^* V}{\partial \varphi} - V \frac{\partial v^*}{\partial \varphi} \right)_{i,j+\frac{1}{2}} & \approx \left\{
    \begin{array}{ll}
      \frac{V_{i,j+\frac{3}{2}}}{2 \Delta{\varphi}} \left( v_{i,j+\frac{1}{2}}^* + v_{i,j+\frac{3}{2}}^* \right) & j = 1 \\
      \frac{1}{2 \Delta{\varphi}} \left[ \left( v_{i,j+\frac{1}{2}}^* + v_{i,j+\frac{3}{2}}^* \right) V_{i,j+\frac{3}{2}} - \left( v_{i,j-\frac{1}{2}}^* + v_{i,j+\frac{1}{2}}^* \right) V_{i,j-\frac{1}{2}} \right] & j = [2,3,\dots,N_\varphi - 2] \\
      - \frac{V_{i,j-\frac{1}{2}}}{2 \Delta{\varphi}} \left( v_{i,j-\frac{1}{2}}^* + v_{i,j+\frac{1}{2}}^* \right) & j = N_\varphi - 1
    \end{array}
  \right. \\
  \left( f U \right)_{i,j+\frac{1}{2}} & \approx \frac{1}{4} \left[ f_{j} \left( U_{i-\frac{1}{2},j} + U_{i+\frac{1}{2},j} \right) + f_{j+1} \left( U_{i-\frac{1}{2},j+1} + U_{i+\frac{1}{2},j+1} \right) \ \right] \quad j = [1, 2, \dots, N_\varphi - 1] \\
  \left(\frac{u \tan{\varphi}}{a} U\right)_{i,j+\frac{1}{2}} & \approx
    \begin{array}{ll}
      \frac{1}{4 a} \left( u_{i+\frac{1}{2},j}^* U_{i+\frac{1}{2},j} + u_{i+\frac{1}{2},j+1}^* U_{i+\frac{1}{2},j+1} \right. \\
      \left. + u_{i-\frac{1}{2},j}^* U_{i-\frac{1}{2},j} + u_{i-\frac{1}{2},j+1}^* U_{i-\frac{1}{2},j+1} \right) & j = [1, 2, \dots, N_\varphi]
    \end{array} \\
  \left( H \frac{\partial \phi^*}{\partial \varphi} \right)_{i,j+\frac{1}{2}} & \approx \frac{H_{i,j} + H_{i,j+1}}{2 \Delta{\varphi}} \left( \phi_{i,j+1}^* - \phi_{i,j}^* \right) \quad j = [1, 2, \dots, N_\varphi - 1]
\end{align}

\subsection{变量$\phi$的作用项}

\begin{align}
  \left( \frac{\partial H U}{\partial \lambda} \right)_{i,j} & \approx \left\{
    \begin{array}{ll}
      0 & j = 1 \\
      \frac{1}{2 \Delta{\lambda}} \left[ \left( H_{i,j} + H_{i+1,j} \right) U_{i+\frac{1}{2},j} - \left( H_{i-1,j} + H_{i,j} \right) U_{i-\frac{1}{2},j} \right] & j = [2, 3, \dots, N_\varphi - 1] \\
      0 & j = N_\varphi
    \end{array}
  \right. \\
  \left( \frac{\partial H V^*}{\partial \varphi} \right)_{i,j} & \approx \left\{
    \begin{array}{ll}
      \frac{1}{2 N_\lambda \Delta{\varphi}} \sum_{i = 1}^{N_\lambda} \left( H_{i,j} + H_{i,j+1} \right) V_{i,j+\frac{1}{2}}^* & j = 1 \\
      \frac{1}{2 \Delta{\varphi}} \left[ \left( H_{i,j} + H_{i,j+1} \right) V_{i,j+\frac{1}{2}}^* - \left( H_{i,j-1} + H_{i,j} \right) V_{i,j-\frac{1}{2}}^* \right] & j = [2, 3, \dots, N_\varphi - 1] \\
      - \frac{1}{2 N_\lambda \Delta{\varphi}} \sum_{i = 1}^{N_\lambda} \left( H_{i,j-1} + H_{i,j} \right) V_{i,j-\frac{1}{2}}^* & j = N_\varphi
    \end{array}
  \right.
\end{align}

\section{时间积分方案}

将出发方程写为如下半离散形式
\begin{align}
  \frac{\partial F}{\partial t} = - L F
\end{align}
其中$L$是具有反对称性的离散差分格式。

\subsection{欧拉中点时间积分方案}

欧拉中点积分方案是一种隐式积分方案
\begin{align}
  \frac{F^{n+1} - F^n}{\Delta{t}} = - L \left( \frac{F^{n+1} + F^n}{2} \right)
\end{align}
方程两边对$F^{n+1} + F^n$做内积,根据$L$的反对称性可知该方案保持平方守恒($\lVert F^{n+1} \rVert^2 = \lVert F^n \rVert^2$)。

\subsection{欧拉前差时间积分方案}

欧拉前差积分方案是一种显式积分方案,并且会导致能量增长
\begin{align}
  \frac{F^{n+1} - F^n}{\Delta{t}} = - L \left( F^n \right)
\end{align}
采用如下迭代计算格式
\begin{align*}
  F_0^{n+1} & = F^n - \tau_0 L F^n \\
  F_1^{n+1} & = F^n - \tau_1 L F_0^{n+1} \\
  F_2^{n+1} & = F^n - \tau_2 L F_1^{n+1} \\
  & \dots \\
  F_k^{n+1} & = F^n - \tau_k L F_{k-1}^{n+1}
\end{align*}
其中$k = 1, 2, \dots, m$,$\tau_k$是待定时间步长。根据平方守恒要求,可以推出(推导过程见附录\ref{app:欧拉前差时间积分方案的守恒改造})第$k$步的时间步长为
\begin{align}
  \tau_k & = 2 \tau_{k-1} \frac{\left( L F_{k-1}^{n+1}, L F_{k-2}^{n+1} \right)}{\lVert L F_{k-1}^{n+1} \rVert^2}
\end{align}
一般取$\tau_{k-1} = \frac{1}{2} \Delta{t}$,即只修改最后一步的时间步长。

\subsection{Runge-Kutta时间积分方案}

\subsection{蛙跳时间积分方案}

\section{快慢过程分裂方案}

\begin{align}
  \frac{\partial F}{\partial t} = - L_f F - L_s F
\end{align}
其中$L_f$是快过程,$L_s$是慢过程。

设算子$L_f F$是为函数形式$G(F)$,$L_s F$为$S(F)$
\begin{align}
  \frac{\partial F}{\partial t} & = - G(F) - S(F) \label{eqn:一阶导数项表达式} \\
  \frac{\partial^2 F}{\partial t^2} & = - G^\prime(F) \frac{\partial F}{\partial t} - S^\prime \frac{\partial F}{\partial t} = G^\prime G + S^\prime S + G^\prime S + S^\prime G \label{eqn:二阶导数项表达式}
\end{align}
根据Taylor展式,并做入下简记
\begin{align*}
  F(t_0) & = (F)^0 \\
  F(t_0 + \Delta{t}) & = (F)^1 \\
  \left. \frac{\partial F}{\partial t} \right|_{(F)^0} & = \left( \frac{\partial F}{\partial t} \right)^0 \\
  \left. \frac{\partial^2 F}{\partial t^2} \right|_{(F)^0} & = \left( \frac{\partial^2 F}{\partial t^2} \right)^0 \\
  G((F)^0) & = (G)^0 \\
  G^\prime((F)^0) & = (G^\prime)^0 \\
  S((F)^0) & = (S)^0 \\
  S^\prime((F)^0) & = (S^\prime)^0
\end{align*}
则$F$在$t = t_0$附近可以展开为
\begin{align}
  (F)^1 = (F)^0 + \Delta{t} \left( \frac{\partial F}{\partial t} \right)^0 + \frac{\Delta{t}^2}{2} \left( \frac{\partial^2 F}{\partial t^2} \right)^0 + \mathcal{O} \left( \Delta{t}^3 \right)
\end{align}
代入(\ref{eqn:一阶导数项表达式})和(\ref{eqn:二阶导数项表达式})可得
\begin{align}
  (F)^1 = (F)^0 - \Delta{t} \left( (G)^0 + (S)^0 \right) + \frac{\Delta{t}^2}{2} \left( (G^\prime)^0 (G)^0 + (S^\prime)^0 (S)^0 + {\color{red} (G^\prime)^0 (S)^0 + (S^\prime)^0 (G)^0} \right) + \mathcal{O} \left( \Delta{t}^3 \right) \label{eqn:算子分裂等式}
\end{align}
算子分裂方案需要与此式相容才是具有二阶精度的方案,其中红色项是需要特别注意的交叉项。

可以采取多种算子分裂模型
\begin{enumerate}
\item 守恒分裂1型(CSP-1):一次快过程 + 一次慢过程
\begin{align}
  \frac{\partial P}{\partial t} & = - L_f P, (P)^0 = (F)^0 \\
  \frac{\partial Q}{\partial t} & = - L_s Q, (Q)^0 = (P)^1
\end{align}
\item 守恒分裂2型(CSP-2):一次快过程 + 两次慢过程
\begin{align}
  \frac{\partial Q}{\partial t} & = - \frac{1}{2} L_s Q, (Q)^0 = (F)^0 \\
  \frac{\partial P}{\partial t} & = - L_f P, (P)^0 = (Q)^1 \\
  \frac{\partial Q}{\partial t} & = - \frac{1}{2} L_s Q, (Q)^0 = (P)^1
\end{align}
\item 改进分裂模型(ISP):一次快过程 + 两次慢过程
\begin{align}
  \frac{\partial P}{\partial t} & = - L_f P - L_s (F)^0, (P)^0 = (F)^0 \\
  \frac{\partial Q}{\partial t} & = - L_s Q + L_s (F)^0, (Q)^0 = (P)^1
\end{align}
\end{enumerate}
其中快过程可以再细分为任意多步。

\subsection{CSP-1分裂模型}
该模型首先将快过程积分一个时间步长:
\begin{align}
  \frac{\partial P}{\partial t} & = - L_f P, (P)^0 = (F)^0
\end{align}
\begin{align*}
  \frac{\partial P}{\partial t} & = - G(Q) \\
  \frac{\partial^2 P}{\partial t^2} & = G^\prime(Q) G(Q)
\end{align*}
得到$t_0 + \Delta{t}$时刻的值
\begin{align}
  (P)^1 & = (P)^0 + \Delta{t} \left( \frac{\partial P}{\partial t} \right)^0 + \frac{\Delta{t}^2}{2} \left( \frac{\partial^2 P}{\partial t^2} \right)^0 + \mathcal{O} \left( \Delta{t}^3 \right) \nonumber \\
  & = (F)^0 - \Delta{t} (G)^0 + \frac{\Delta{t}^2}{2} (G^\prime)^0 (G)^0 + \mathcal{O} \left( \Delta{t}^3 \right) \label{eqn:CSP-1中P1表达式}
\end{align}
然后以$(P)^1$为初值将慢过程积分一个时间步:
\begin{align}
  \frac{\partial Q}{\partial t} = - L_s Q, (Q)^0 = (P)^1
\end{align}
\begin{align*}
  \frac{\partial Q}{\partial t} & = - S(Q) \\
  \frac{\partial^2 Q}{\partial t^2} & = S^\prime(Q) S(Q)
\end{align*}
得到$t_0 + \Delta{t}$时刻的最终值
\begin{align}
  (F)^1 & = (Q)^0 + \Delta{t} \left( \frac{\partial Q}{\partial t} \right)^0 + \frac{\Delta{t}^2}{2} \left( \frac{\partial^2 Q}{\partial t^2} \right)^0 + \mathcal{O} \left( \Delta{t}^3 \right) \nonumber \\
  & = (P)^1 - \Delta{t} {\color{red}S\left((P)^1\right)} + \frac{\Delta{t}^2}{2} {\color{blue}S^\prime\left((P)^1\right)} {\color{red}S\left((P)^1\right)} + \mathcal{O} \left( \Delta{t}^3 \right)
\end{align}
其中红蓝项体现了快过程对慢过程的影响,代入$(P)^1$表达式(\ref{eqn:CSP-1中P1表达式}),并以$(F)^0$做Taylor展开
\begin{align}
  {\color{red}S\left((P)^1\right)} & = S\left( (F)^0 - \Delta{t} (G)^0 + \frac{\Delta{t}^2}{2} (G^\prime)^0 (G)^0 + \mathcal{O} \left( \Delta{t}^3 \right) \right) \nonumber \\
  & = (S)^0 + (S^\prime)^0 \left( - \Delta{t} (G)^0 + \frac{\Delta{t}^2}{2} (G^\prime)^0 (G)^0 + \mathcal{O} \left( \Delta{t}^3 \right) \right) + \mathcal{O} \left( \Delta{t}^2 \right) \nonumber \\
  & = {\color{red}(S)^0 - \Delta{t} (S^\prime)^0 (G)^0 + \mathcal{O} \left( \Delta{t}^2 \right)} \label{eqn:CSP-1中的SP} \\
  {\color{blue}S^\prime\left((P)^1\right)} & = S^\prime \left( (F)^0 - \Delta{t} (G)^0 + \frac{\Delta{t}^2}{2} (G^\prime)^0 (G)^0 + \mathcal{O} \left( \Delta{t}^3 \right) \right) \nonumber \\
  & = {\color{blue}(S^\prime)^0 + \mathcal{O} \left( \Delta{t} \right)} \label{eqn:CSP-1中的S'P}
\end{align}
将(\ref{eqn:CSP-1中的SP})和(\ref{eqn:CSP-1中的S'P}),以及(\ref{eqn:CSP-1中P1表达式})代入上式得到
\begin{align}
  (F)^1 & = (F)^0 - \Delta{t} (G)^0 + \frac{\Delta{t}^2}{2} (G^\prime)^0 (G)^0 + \mathcal{O} \left( \Delta{t}^3 \right) \nonumber \\
  & - \Delta{t} \left( {\color{red}(S)^0 - \Delta{t} (S^\prime)^0 (G)^0 + \mathcal{O} \left( \Delta{t}^2 \right)} \right) \nonumber \\
  & + \frac{\Delta{t}^2}{2} \left( {\color{blue}(S^\prime)^0 + \mathcal{O} \left( \Delta{t} \right)} \right) \left( {\color{red}(S)^0 - \Delta{t} (S^\prime)^0 (G)^0 + \mathcal{O} \left( \Delta{t}^2 \right)} \right) + \mathcal{O} \left( \Delta{t}^3 \right) \nonumber \\
  & = (F)^0 - \Delta{t} \left( (G)^0 + (S)^0 \right) + \frac{\Delta{t}^2}{2} \left( (G^\prime)^0 (G)^0 + (S^\prime)^0 (S)^0 \right) + \Delta{t}^2 (S^\prime)^0 (G)^0 + \mathcal{O} \left( \Delta{t}^3 \right)
\end{align}
与(\ref{eqn:算子分裂等式})相比较,虽然是相容的,但是算子交叉项不对应,因此时间精度是一阶。

\subsection{CSP-2分裂模型}
该模型首先以一半慢过程作用积分一个时间步长:
\begin{align}
  \frac{\partial Q}{\partial t} = - \frac{1}{2} L_s Q, (Q)^0 = (F)^0
\end{align}
\begin{align*}
  \frac{\partial Q}{\partial t} & = - \frac{1}{2} S(Q) \\
  \frac{\partial^2 Q}{\partial t^2} & = \frac{1}{4} S^\prime(Q) S(Q)
\end{align*}
得到$t_0 + \Delta{t}$时刻的值
\begin{align}
  (Q)^1 & = (Q)^0 + \Delta{t} \left( \frac{\partial Q}{\partial t} \right)^0 + \frac{\Delta{t}^2}{2} \left( \frac{\partial^2 Q}{\partial t^2} \right)^0 + \mathcal{O} \left( \Delta{t}^3 \right) \nonumber \\
  & = (F)^0 - \frac{\Delta{t}}{2} (S)^0 + \frac{\Delta{t}^2}{8} (S^\prime)^0 (S)^0 + \mathcal{O} \left( \Delta{t}^3 \right) \label{eqn:CSP-2中Q1表达式}
\end{align}
然后以$(Q)^1$为初值将快过程积分一个时间步:
\begin{align}
  \frac{\partial P}{\partial t} = - L_f P, (P)^0 = (Q)^1
\end{align}
\begin{align*}
  \frac{\partial P}{\partial t} & = - G(P) \\
  \frac{\partial^2 P}{\partial t^2} & = G^\prime(P) G(P)
\end{align*}
得到$t_0 + \Delta{t}$时刻的值
\begin{align}
  (P)^1 & = (P)^0 + \Delta{t} \left( \frac{\partial P}{\partial t} \right)^0 + \frac{\Delta{t}^2}{2} \left( \frac{\partial^2 P}{\partial t^2} \right)^0 + \mathcal{O} \left( \Delta{t}^3 \right) \nonumber \\
  & = (Q)^1 - \Delta{t} {\color{red}G\left((Q)^1\right)} + \frac{\Delta{t}^2}{2} {\color{blue}G^\prime\left((Q)^1\right)} {\color{red}G\left((Q)^1\right)} + \mathcal{O} \left( \Delta{t}^3 \right) \nonumber
\end{align}
其中红蓝项体现了慢过程对快过程的影响,代入$(Q)^1$表达式(\ref{eqn:CSP-2中Q1表达式}),并以$(F)^0$做Taylor展开
\begin{align}
  {\color{red}G\left((Q)^1\right)} & = G\left( (F)^0 - \frac{\Delta{t}}{2} (S)^0 + \frac{\Delta{t}^2}{8} (S^\prime)^0 (S)^0 + \mathcal{O} \left( \Delta{t}^3 \right) \right) \nonumber \\
  & = G\left((F)^0\right) + G^\prime\left((F)^0\right) \left( - \frac{\Delta{t}}{2} (S)^0 {\color{gray} + \frac{\Delta{t}^2}{8} (S^\prime)^0 (S)^0 + \mathcal{O} \left( \Delta{t}^3 \right)} \right) + \mathcal{O} \left( \Delta{t}^2 \right) \nonumber \\
  & = (G)^0 - \frac{\Delta{t}}{2} (G^\prime)^0 (S)^0 + \mathcal{O} \left( \Delta{t}^2 \right) \label{eqn:CSP-2中的GQ} \\
  {\color{blue}G^\prime\left((Q)^1\right)} & = G^\prime \left( (F)^0 - \frac{\Delta{t}}{2} (S)^0 + \frac{\Delta{t}^2}{8} (S^\prime)^0 (S)^0 + \mathcal{O} \left( \Delta{t}^3 \right) \right) \nonumber \\
  & = (G^\prime)^0 + \mathcal{O} \left( \Delta{t} \right) \label{eqn:CSP-2中的G'Q}
\end{align}
将(\ref{eqn:CSP-2中的GQ})和(\ref{eqn:CSP-2中的G'Q}),以及(\ref{eqn:CSP-2中Q1表达式})代入上式得到
\begin{align}
  (P)^1 = (F)^0 - \frac{\Delta{t}}{2} (S)^0 + \frac{\Delta{t}^2}{8} (S^\prime)^0 (S)^0 - \Delta{t} (G)^0 + \frac{\Delta{t}^2}{2} (G^\prime)^0 (S)^0  + \frac{\Delta{t}^2}{2} (G^\prime)^0 (G)^0 + \mathcal{O} \left( \Delta{t}^3 \right) \label{eqn:CSP-2中P1表达式}
\end{align}
最后以$(P)^1$为初值以一半慢过程作用积分一个时间步:
\begin{align}
  \frac{\partial Q}{\partial t} = - \frac{1}{2} L_s Q, (Q)^0 = (P)^1
\end{align}
\begin{align*}
  \frac{\partial Q}{\partial t} & = - \frac{1}{2} S(Q) \\
  \frac{\partial^2 Q}{\partial t^2} & = \frac{1}{4} S^\prime(Q) S(Q)
\end{align*}
得到$t_0 + \Delta{t}$时刻的最终值
\begin{align}
  (F)^1 & = (Q)^0 + \Delta{t} \left( \frac{\partial Q}{\partial t} \right)^0 + \frac{\Delta{t}^2}{2} \left( \frac{\partial^2 Q}{\partial t^2} \right)^0 + \mathcal{O} \left( \Delta{t}^3 \right) \nonumber \\
  & = {\color{green}(P)^1} - \frac{\Delta{t}}{2} {\color{red}S\left((P)^1\right)} + \frac{\Delta{t}^2}{8} {\color{blue}S^\prime\left((P)^1\right)} {\color{red}S\left((P)^1\right)} + \mathcal{O} \left( \Delta{t}^3 \right)
\end{align}
其中红蓝项体现了快过程对慢过程的影响,代入$(P)^1$的表示式(\ref{eqn:CSP-2中P1表达式}),并以$(F)^0$做Taylor展开
\begin{align}
  {\color{red}S\left((P)^1\right)} & = S\left( (F)^0 - \frac{\Delta{t}}{2} (S)^0 - \Delta{t} (G)^0 {\color{gray}+ \frac{\Delta{t}^2}{8} (S^\prime)^0 (S)^0 + \frac{\Delta{t}^2}{2} (G^\prime)^0 (S)^0  + \frac{\Delta{t}^2}{2} (G^\prime)^0 (G)^0 + \mathcal{O} \left( \Delta{t}^3 \right)} \right) \nonumber \\
  & = (S)^0 + (S^\prime)^0 \left( - \frac{\Delta{t}}{2} (S)^0 - \Delta{t} (G)^0 + {\color{gray}\mathcal{O} \left( \Delta{t}^2 \right)} \right) + \mathcal{O} \left( \Delta{t}^2 \right) \nonumber \\
  & = {\color{red}(S)^0 - \frac{\Delta{t}}{2} (S^\prime)^0 (S)^0 - \Delta{t} (S^\prime)^0 (G)^0 + \mathcal{O} \left( \Delta{t}^2 \right)} \label{eqn:CSP-2中的SP} \\
  {\color{blue}S^\prime\left((P)^1\right)} & = S^\prime\left( (F)^0 {\color{gray}- \frac{\Delta{t}}{2} (S)^0 - \Delta{t} (G)^0 + \frac{\Delta{t}^2}{8} (S^\prime)^0 (S)^0 + \frac{\Delta{t}^2}{2} (G^\prime)^0 (S)^0  + \frac{\Delta{t}^2}{2} (G^\prime)^0 (G)^0 + \mathcal{O} \left( \Delta{t}^3 \right)} \right) \nonumber \\
  & = {\color{blue}(S^\prime)^0 + \mathcal{O} \left( \Delta{t} \right)} \label{eqn:CSP-2中的S'P}
\end{align}
将(\ref{eqn:CSP-2中的SP})和(\ref{eqn:CSP-2中的S'P}),以及(\ref{eqn:CSP-2中P1表达式})代入上式得到
\begin{align}
  (F)^1 & = {\color{green}(F)^0 - \frac{\Delta{t}}{2} (S)^0 + \frac{\Delta{t}^2}{8} (S^\prime)^0 (S)^0 - \Delta{t} (G)^0 + \frac{\Delta{t}^2}{2} (G^\prime)^0 (S)^0 + \frac{\Delta{t}^2}{2} (G^\prime)^0 (G)^0 + \mathcal{O} \left( \Delta{t}^3 \right)} \nonumber \\
  & {\color{red}- \frac{\Delta{t}}{2} (S)^0 + \frac{\Delta{t}^2}{4} (S^\prime)^0 (S)^0 + \frac{\Delta{t}^2}{2} (S^\prime)^0 (G)^0 + \mathcal{O} \left( \Delta{t}^3 \right)} \nonumber \\
  & + \frac{\Delta{t}^2}{8} \left( {\color{blue}(S^\prime)^0 + \mathcal{O} \left( \Delta{t} \right)} \right) \left( {\color{red}(S)^0} {\color{gray}- \frac{\Delta{t}}{2} (S^\prime)^0 (S)^0 - \Delta{t} (S^\prime)^0 (G)^0 + \mathcal{O} \left( \Delta{t}^2 \right)} \right) \nonumber \\
  & = (F)^0 - \Delta{t} \left( (S)^0 + (G)^0 \right) + \frac{\Delta{t}^2}{2} \left( (S^\prime)^0 (S)^0 + (G^\prime)^0 (G)^0 + (G^\prime)^0 (S)^0 + (S^\prime)^0 (G)^0 \right) + \mathcal{O} \left( \Delta{t^3} \right) \nonumber
\end{align}
上式与(\ref{eqn:算子分裂等式})相容,且具有二阶时间精度。

\begin{appendix}

\section{平流项的反对称性}
\label{app:平流项的反对称性}
考察如下两种等价连续形式在Arakawa-C网格下离散格式的反对称性
\begin{align}
  2 \frac{\partial u U}{\partial \lambda} - U \frac{\partial u}{\partial \lambda} + 2 \frac{\partial v \cos{\varphi} U}{\partial \varphi} - U \frac{\partial v \cos{\varphi}}{\partial \varphi} \\
  \frac{\partial u U}{\partial \lambda} + u \frac{\partial U}{\partial \lambda} + \frac{\partial v \cos{\varphi} U}{\partial \varphi} + v \cos{\varphi} \frac{\partial U}{\partial \varphi}
\end{align}
\begin{align}
  \frac{\partial u U}{\partial \lambda} & = \frac{1}{4 \Delta{\lambda}} \left[ \left( u_{i+\frac{1}{2}} + u_{i+\frac{3}{2}} \right) \left( U_{i+\frac{1}{2}} + U_{i+\frac{3}{2}} \right) - \left( u_{i+\frac{1}{2}} + u_{i-\frac{1}{2}} \right) \left( U_{i+\frac{1}{2}} + U_{i-\frac{1}{2}} \right) \right] \nonumber \\
  U \frac{\partial u}{\partial \lambda} & = \frac{U_{i+\frac{1}{2}}}{2 \Delta{\lambda}} \left[ \left( u_{i+\frac{1}{2}} + u_{i+\frac{3}{2}} \right) - \left( u_{i+\frac{1}{2}} + u_{i-\frac{1}{2}} \right) \right] = \frac{U_{i+\frac{1}{2}}}{2 \Delta{\lambda}} \left[ u_{i+\frac{3}{2}} - u_{i-\frac{1}{2}} \right] \nonumber \\
  2 \frac{\partial u U}{\partial \lambda} - U \frac{\partial u}{\partial \lambda} & = \frac{1}{2 \Delta{\lambda}} \left[ \left( u_{i+\frac{1}{2}} + u_{i+\frac{3}{2}} \right) \left( U_{i+\frac{1}{2}} + U_{i+\frac{3}{2}} \right) - \left( u_{i+\frac{1}{2}} + u_{i-\frac{1}{2}} \right) \left( U_{i+\frac{1}{2}} + U_{i-\frac{1}{2}} \right) \right. \nonumber \\
  & \quad\quad\quad \left. - u_{i+\frac{3}{2}} U_{i+\frac{1}{2}} + u_{i-\frac{1}{2}} U_{i+\frac{1}{2}} \right] \nonumber \\
  & = \frac{1}{2 \Delta{\lambda}} \left[ \left( u_{i+\frac{1}{2}} + u_{i+\frac{3}{2}} \right) U_{i+\frac{3}{2}} - \left( u_{i+\frac{1}{2}} + u_{i-\frac{1}{2}} \right) U_{i-\frac{1}{2}} \right] \label{eqn:平流项第一种形式离散格式}
\end{align}
\begin{align}
  \frac{\partial u U}{\partial \lambda} & = \frac{1}{4 \Delta{\lambda}} \left[ \left( u_{i+\frac{1}{2}} + u_{i+\frac{3}{2}} \right) \left( U_{i+\frac{1}{2}} + U_{i+\frac{3}{2}} \right) - \left( u_{i+\frac{1}{2}} + u_{i-\frac{1}{2}} \right) \left( U_{i+\frac{1}{2}} + U_{i-\frac{1}{2}} \right) \right] \nonumber \\
  u \frac{\partial U}{\partial \lambda} & = \frac{u_{i+\frac{1}{2}}}{2 \Delta{\lambda}} \left[ \left( U_{i+\frac{1}{2}} + U_{i+\frac{3}{2}} \right) - \left( U_{i+\frac{1}{2}} + U_{i-\frac{1}{2}} \right) \right] = \frac{u_{i+\frac{1}{2}}}{2 \Delta{\lambda}} \left[ U_{i+\frac{3}{2}} - U_{i-\frac{1}{2}} \right] \nonumber \\
  \frac{\partial u U}{\partial \lambda} + u \frac{\partial U}{\partial \lambda} & = \frac{1}{2 \Delta{\lambda}} \left[ \frac{\left( u_{i+\frac{1}{2}} + u_{i+\frac{3}{2}} \right) \left( U_{i+\frac{1}{2}} + U_{i+\frac{3}{2}} \right)}{2} - \frac{\left( u_{i+\frac{1}{2}} + u_{i-\frac{1}{2}} \right) \left( U_{i+\frac{1}{2}} + U_{i-\frac{1}{2}} \right)}{2} \right. \nonumber \\
  & \quad\quad\quad \left. - u_{i+\frac{1}{2}} U_{i+\frac{3}{2}} + u_{i+\frac{1}{2}} U_{i-\frac{1}{2}} \right] \nonumber \\
  & \quad\quad\quad\quad\quad\quad  \frac{1}{2} \left( u_{i+\frac{1}{2}} + u_{i+\frac{3}{2}} - u_{i+\frac{1}{2}} - u_{i-\frac{1}{2}} \right) U_{i+\frac{1}{2}} \nonumber \\{}
  & \quad\quad\quad\quad\quad\quad  \frac{1}{2} \left( u_{i+\frac{1}{2}} + u_{i+\frac{3}{2}} - 2 u_{i+\frac{1}{2}} \right) U_{i+\frac{3}{2}} \nonumber \\
  & \quad\quad\quad\quad\quad\quad  - \frac{1}{2} \left( u_{i+\frac{1}{2}} + u_{i-\frac{1}{2}} - 2 u_{i+\frac{1}{2}} \right) U_{i-\frac{1}{2}} \nonumber \\
  & = \frac{1}{4 \Delta{\lambda}} \left[ \left( u_{i+\frac{3}{2}} - u_{i+\frac{1}{2}} \right) U_{i+\frac{3}{2}} + {\color{red}\left( u_{i+\frac{3}{2}} - u_{i-\frac{1}{2}} \right) U_{i+\frac{1}{2}}} + \left( u_{i+\frac{1}{2}} - u_{i-\frac{1}{2}} \right) U_{i-\frac{1}{2}} \right] \label{eqn:平流项第二种形式离散格式}
\end{align}
由此可见(\ref{eqn:平流项第一种形式离散格式})可以保证反对称性,而(\ref{eqn:平流项第二种形式离散格式})多了$U_{i+\frac{1}{2}}$项,无法满足反对称性要求。

\section{欧拉前差时间积分方案的守恒改造}
\label{app:欧拉前差时间积分方案的守恒改造}

目标是找出待定时间步长$\tau_k$使得格式满足平方守恒。抽出某两个相邻迭代步
\begin{align}
  F_{k-1}^{n+1} & = F^n - \tau_{k-1} L F_{k-2}^{n+1} \\
  F_k^{n+1} & = F^n - \tau_k L F_{k-1}^{n+1} \label{eqn:欧拉前差迭代终步方程}
\end{align}
将方程两边与$L F_{k-1}^{n+1}$做内积
\begin{align*}
  \left( L F_{k-1}^{n+1}, F_{k-1}^{n+1} \right) & = \left( L F_{k-1}^{n+1}, F^n \right) - \tau_{k-1} \left( L F_{k-1}^{n+1}, L F_{k-2}^{n+1} \right) \\
  \left( L F_{k-1}^{n+1}, F_k^{n+1} \right) & = \left( L F_{k-1}^{n+1}, F^n \right) - \tau_k \lVert L F_{k-1}^{n+1} \rVert^2
\end{align*}
由于$L$具有反对称性,上两式可化为
\begin{align}
  {\color{blue}\left( L F_{k-1}^{n+1}, F^n \right)} & = {\color{green}\tau_{k-1} \left( L F_{k-1}^{n+1}, L F_{k-2}^{n+1} \right)} \nonumber \\
  {\color{red}\left( L F_{k-1}^{n+1}, F_k^{n+1} \right)} & = {\color{green}\tau_{k-1} \left( L F_{k-1}^{n+1}, L F_{k-2}^{n+1} \right)} - \tau_k \lVert L F_{k-1}^{n+1} \rVert^2 \label{eqn:欧拉前差推导中间式子}
\end{align}
再根据$\lVert F_k^{n+1} \rVert^2 = \lVert F^n \rVert^2$的目标条件,将(\ref{eqn:欧拉前差迭代终步方程})两端与$F_k^{n+1} + F^n$做内积
\begin{align*}
  \lVert F_k^{n+1} \rVert^2 - \lVert F^n \rVert^2 & = - \tau_k \left( L F_{k-1}^{n+1}, F_k^{n+1} + F^n \right) \\
  0 & = \left( L F_{k-1}^{n+1}, F_k^{n+1} + F^n \right) \\
  {\color{red}\left( L F_{k-1}^{n+1}, F_k^{n+1} \right)} & = - {\color{blue}\left( L F_{k-1}^{n+1}, F^n \right)} \\
  & = - {\color{green}\tau_{k-1} \left( L F_{k-1}^{n+1}, L F_{k-2}^{n+1} \right)}
\end{align*}
代入到(\ref{eqn:欧拉前差推导中间式子})可得
\begin{align*}
  \tau_k & = 2 \tau_{k-1} \frac{\left( L F_{k-1}^{n+1}, L F_{k-2}^{n+1} \right)}{\lVert L F_{k-1}^{n+1} \rVert^2}
\end{align*}

\end{appendix}

\end{document}